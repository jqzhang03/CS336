\documentclass[a4paper,10pt]{ctexart} 

% ============================================================
%   模块 A: 页面布局与基本设置
% ============================================================
\usepackage[left=2.5cm, right=2.5cm, top=2.5cm, bottom=2.5cm]{geometry} % 页边距
\usepackage{fancyhdr} % 页眉页脚控制
\usepackage{lastpage} % 获取总页数
\usepackage{hyperref} % 超链接与书签
\usepackage{enumitem} % 定义列表
\hypersetup{
    colorlinks=true,
    linkcolor=blue,
    filecolor=magenta,      
    urlcolor=cyan,
}

% ============================================================
%   模块 B: 数学支持
% ============================================================
\usepackage{amsmath, amssymb, amsthm} % 数学公式、符号、定理
\usepackage{mathrsfs} % 花体字母 \mathscr

% ============================================================
%   模块 C: 图形与代码高亮
% ============================================================
\usepackage{graphicx} % 插入图片
\usepackage{float}    % 强制图片位置 [H]
\usepackage{listings} % 代码插入
\usepackage{xcolor}   % 颜色支持
\usepackage{lstautogobble}
% 代码块样式设置
\definecolor{codegray}{rgb}{0.95,0.95,0.95}
\definecolor{codeblue}{rgb}{0,0,0.6}
\lstset{
    backgroundcolor=\color{codegray},
    commentstyle=\color{green!60!black},
    keywordstyle=\color{codeblue},
    numberstyle=\tiny\color{gray},
    stringstyle=\color{purple},
    basicstyle=\ttfamily\small,
    autogobble=true,
    breakatwhitespace=false,         
    breaklines=true,                 
    captionpos=b,                    
    keepspaces=true,                 
    numbers=left,                    
    numbersep=5pt,                  
    showspaces=false,                
    showstringspaces=false,
    showtabs=false,                  
    tabsize=4,
    frame=single
}

% ============================================================
%   模块 D: 美化框 (tcolorbox) - 核心部分
% ============================================================
\usepackage[most]{tcolorbox} % 强大的盒子宏包

% --- 定义 1: 概念/定义框 (绿色风格) ---
\newtcolorbox[auto counter, number within=section]{definition}[2][]{
    colback=green!5!white,      % 背景色
    colframe=green!40!black,    % 边框色
    coltitle=white,             % 标题文字颜色
    boxrule=1pt,
    title=\textbf{定义~\thetcbcounter: #2}, % 标题格式
    fonttitle=\bfseries,
    breakable,                  % 允许跨页
    enhanced,
    attach boxed title to top left={yshift=-2mm, xshift=2mm}, % 标题位置
    #1
}

% --- 定义 2: 定理/结论框 (蓝色风格) ---
\newtcolorbox[auto counter, number within=section]{theorem}[2][]{
    colback=blue!5!white,
    colframe=blue!50!black,
    coltitle=white,
    boxrule=1pt,
    title=\textbf{定理~\thetcbcounter: #2},
    fonttitle=\bfseries,
    breakable,
    enhanced,
    attach boxed title to top left={yshift=-2mm, xshift=2mm},
    #1
}

% --- 定义 3: 注意/警告框 (红色风格) ---
\newtcolorbox{alert}{
    colback=red!5!white,
    colframe=red!75!black,
    title=\textbf{注意},
    fonttitle=\bfseries,
    breakable,
    enhanced
}

\newtcolorbox{problembox}{
    colframe=orange,        % 边框橙色
    colback=white,          % 背景白色
    boxrule=2pt,            % 边框粗细
    sharp corners,          % 直角
    breakable,              % 允许跨页
    left=5pt, right=5pt,  % 内部左右留白
    top=5pt, bottom=5pt   % 内部上下留白
}

% ============================================================
%   页面头部信息设置
% ============================================================
\pagestyle{fancy}
\fancyhf{}
\lhead{\textbf{课程名称:CS336}} % 左页眉
\rhead{日期:2026年1月21日}           % 右页眉
\cfoot{第 \thepage \ 页,共 \pageref{LastPage} 页} % 页脚
\renewcommand{\headrulewidth}{0.4pt} % 页眉分割线

% ==================== 4. 自定义题目样式命令 ====================
% 用法: \problem{标题内容}
\newcommand{\problem}[1]{%
    \par\vspace{1em}
    \noindent{\large\bfseries #1} % 大号加粗字体
    \par\vspace{2pt}
    \hrule height 0.8pt           % 下划线,厚度0.8pt
    \vspace{1em}
}

% 定义 "Deliverable" 的快捷命令
\newcommand{\deliverable}[1]{%
    \par\vspace{3pt}
    \noindent \textbf{Deliverable:} #1
}


% ============================================================
%   文档正文开始
% ============================================================
\title{\textbf{Lab1}}
\date{2026年1月21日}

\begin{document}

    \maketitle % 生成标题
    \tableofcontents % 生成目录
    \newpage % 换页

    % --- 正文内容 ---

    \section{基本概念}
    \subsection{Tokenizer}
    \subsubsection{Word-level tokenizer}
    将每个单词定义为一个token,这样相较于Byte-level tokenizer来说,对于一个长句子,会产生更少的token,但由于需要为每个词都定义一个序号,必然会存在某些词并不在定义的字典当中,从而无法解析
    \subsubsection{Byte-level tokenizer}
    将每一个字符定义为一个token,其字典大小为0-255,这样可以覆盖到每一个可见的字母,但缺点是如果一个单词的长度很长,会使得token值特别大,在模型训练的时候会更浪费时间
    \subsubsection{Subword tokenizer(BPE)}
    结合Word-level tokenizer和Byte-level tokenizer的优点,得到了BPE这种压缩算法,即Byte-pair encoding

    对于一个句子,我们的字典最初就是Byte-level tokenizer的字典,序号是0-255,然后遍历整个句子,每次将相邻的两个单词组成一个pair,然后记录该pair出现的次数,遍历完以后,将出现次数最高的pair定义为一个新的字典序号,
    并加入到字典当中,然后将该pair看成是一个word,持续遍历整个句子,直到合并完成。直接做的缺点有每次我们都需要完整遍历语料库,计算量非常大,其次,如果是相邻的标点符号不一样,也会生成一个新的ID,尽管标点前面
    的单词是一样的

    在最初提出的BPE中,字典里存在一些\texttt{Special tokens},比如\texttt{<|endoftext|>},其用途是告诉程序什么时候停止生成token,\texttt{special tokens}不会被分割成多个token,在该实验的BPE中不考虑special token

    给定\texttt{low}、\texttt{lower}、\texttt{widest}、\texttt{newest},根据BPE的步骤进行分词
    \begin{figure}[H]
        \centering
        \includegraphics[width=\textwidth]{images/FB5BC6A57B760A9C422BFD060B71D22A.jpg}
    \end{figure}

    下面是一个\href{https://tiktokenizer.vercel.app/?encoder=gpt2}{Token count}网页,可以在网页内选择不同的模型,看相同的一个句子其生成的token的方式的不同

    \section{问题部分}

    \subsection{Problem unicode1}
    \begin{problembox}
        % --- 题目头 ---
        \problem{Problem (unicode1): Understanding Unicode (1 point)}

        % --- 列表开始 ---
        \begin{enumerate}[label=(\alph*), leftmargin=*]

            % (a)
            \item What Unicode character does \texttt{chr(0)} return?
            
            \deliverable{A one-sentence response.}

            % (b)
            \item How does this character's string representation (\texttt{\_\_repr\_\_()}) differ from its printed representation?
            
            \deliverable{A one-sentence response.}

            % (c)
            \item What happens when this character occurs in text? It may be helpful to play around with the following in your Python interpreter and see if it matches your expectations:

            % 代码块
            \begin{lstlisting}
                >>> chr(0)
                >>> print(chr(0))
                >>> "this is a test" + chr(0) + "string"
                >>> print("this is a test" + chr(0) + "string")
            \end{lstlisting}

            \deliverable{A one-sentence response.}

        \end{enumerate}
    \end{problembox}
    \subsubsection{Answer}
    \begin{enumerate}[label=(\alph*), leftmargin=*]
        \item 在终端中使用python环境进行实现 \\
        \begin{figure}[H] % [H] 表示"Here",强制让图片就在当前代码位置,不乱跑
            \centering
            % width=0.8\textwidth 表示图片宽度占页面文本宽度的 80%
            \includegraphics[width=0.2\textwidth]{./images/QQ20260124-113852.png}
        \end{figure}
        \item \texttt{chr(0)}是空字符,直接在命令行输入命令或调用\texttt{\_\_repr\_\_()}时会用转义字符来表示,对于非打印字符来说直接打印是不可见的\\
        \begin{figure}[H]
            \centering
            \includegraphics[width=0.3\textwidth]{images/QQ20260124-115248.png} 
        \end{figure}
        \item \texttt{print()}是位于用户视图,直接在交互式命令行输入命令或调用\texttt{\_\_repr\_\_()}是位于开发者视图,因此一个可见空字符,一个不可见空字符
        \begin{figure}[H]
            \centering
            \includegraphics[width=0.8\textwidth]{images/QQ20260124-115550.png} 
        \end{figure}
    \end{enumerate}

    \subsection{Problem unicode2}
    \begin{problembox}
        \problem{Problem (unicode2): Unicode Encodings \quad (3 points)}

        \begin{enumerate}[label=(\alph*), leftmargin=*]

            \item What are some reasons to prefer training our tokenizer on UTF-8 encoded bytes, rather than UTF-16 or UTF-32? It may be helpful to compare the output of these encodings for various input strings.

            \deliverable{A one-to-two sentence response.}

            \item Consider the following (incorrect) function, which is intended to decode a UTF-8 byte string into a Unicode string. Why is this function incorrect? Provide an example of an input byte string that yields incorrect results.

            \begin{lstlisting}
                def decode_utf8_bytes_to_str_wrong(bytestring: bytes):
                    return "".join([bytes([b]).decode("utf-8") for b in bytestring])

                >>> decode_utf8_bytes_to_str_wrong("hello".encode("utf-8"))
                'hello'
            \end{lstlisting}

            \deliverable{An example input byte string for which \texttt{decode\_utf8\_bytes\_to\_str\_wrong} produces incorrect output, with a one-sentence explanation of why the function is incorrect.}

            \item Give a two byte sequence that does not decode to any Unicode character(s).

            \deliverable{An example, with a one-sentence explanation.}

        \end{enumerate}

    \end{problembox}

    \subsubsection{Answer}
    \begin{enumerate}[label=(\alph*), leftmargin=*]
        \item UTF-8相较于UTF-16和UTF-32有以下几个优点:
        \begin{itemize}
            \item UTF-8完全兼容ASCII码,而UTF-16和UTF-32处理ASCII字符时会插入大量空字节
            \item UTF-8是变长编码,通常是1到4个字节不等,对比UTF-16和UTF-32更节省空间
            \item UTF-8是以单字节为单位处理的,不存在大端序和小端序转换的问题,而UTF-16和UTF-32必须处理字节序问题,否则会出现乱码
            \item 如果文件在传输过程中丢失了一个字节或受损,UTF-8能根据后续的字节特征快速找到下一个字符的起始位置
        \end{itemize}
        \item 这段代码会出错,因为在函数当中处理bytestring的方式是一个字节一个字节的解码,但UTF-8是变长编码,如果输入的是多个字节组成的字符会导致\textbf{UnicodeDecodeError}
        \begin{figure}[H]
            \centering
            \includegraphics[width=0.8\textwidth]{images/QQ20260124-222934.png}
            \includegraphics[width=0.8\textwidth]{images/QQ20260124-222943.png} 
        \end{figure}
        \item \texttt{0XC0 0XAF},在现代解码器中任何以\texttt{0XC0}、\texttt{0XC1}开头的字节序列都会被抛出\textbf{UnicodeDecodeError}
    \end{enumerate}

    \subsection{Problem train\_bpe}
    \begin{figure}[H]
        \centering
        \includegraphics[width=\textwidth]{images/QQ20260126-213522.png}
    \end{figure}

    \subsubsection{Answer}
    根据问题内容,我们首先应该在\textbf{./tests/adapters.py}完善其\textbf{run\_train\_bpe}函数,打开文件后,我们不难发现整个函数只有一句
    \begin{lstlisting}
        raise NotImplementedError
    \end{lstlisting}
    其主要目的就是告诉我们的测试程序应该去哪里找到我们自己编写的函数,以及传入和接收的一些参数值
    \begin{lstlisting}
        from cs336_basics.train_bpe import train_bpe
        return train_bpe(
            input_path = input_path,
            vocab_size = vocab_size,
            special_tokens = special_tokens
        )
    \end{lstlisting}
    首先\textbf{./cs336\_basics/train\_bpe.py}为无任何优化的原始版本,代码中有较为详细的注释,执行以下命令测试代码,警告无需在意
    \begin{lstlisting}
        uv run pytest tests/test_train_bpe.py
    \end{lstlisting}
    \begin{figure}[H]
        \centering
        \includegraphics[width=\textwidth]{images/QQ20260131-153735.png}
    \end{figure}
    接下来对无任何优化的分词器根据文档中的提示进行优化,代码见\textbf{./cs336\_basics/train\_bpe\_
    parallel.py},详细注释见代码,
    需要注意的是我们在并行运算的代码中需要将文本进行切块,就需要用到\textbf{./cs336
    \_basics/pretokenization\_example.py}中的函数,我们需要先给该代码添加一个main函数,
    否则运行时会报错
    \begin{lstlisting}
        ## Usage
        if __name__ == "__main__" :
            with open(..., "rb") as f:
                num_processes = 4
                boundaries = find_chunk_boundaries(f, num_processes, b"<|endoftext|>")
    \end{lstlisting}
    接着修改好\textbf{adapters.py}中调用的函数后,依然是执行
    \begin{lstlisting}
        uv run pytest tests/test_train_bpe.py
    \end{lstlisting}
    不难发现,速度的改变还是非常显著的
    \begin{figure}[H]
        \centering
        \includegraphics[width=\textwidth]{images/QQ20260131-162403.png}
    \end{figure}

    \subsection{Problem train\_bpe\_tinystories}
    \begin{figure}[H]
        \centering
        \includegraphics[width=\textwidth]{images/QQ20260130-110300.png}
    \end{figure}

    \subsubsection{Answer}
    \begin{itemize}
        \item 首先利用原生的BPE分词器进行实验,即不加任何优化和多线程等方式,然后在终端利用
        \begin{lstlisting}
            uv run python -m cProfile -o workspace/tinystories.prof cs336_basics/train_bpe_tinystories.py
        \end{lstlisting}
        得到如下结果
        \begin{figure}[H]
            \centering
            \includegraphics[width=\textwidth]{images/QQ20260130-105328.png}
        \end{figure}
        在删减版训练集上用时24.54s,占用0.05GB内存,最长的token是' congratulations',有前导空格
        \item 在终端利用
        \begin{lstlisting}
            uv run python - <<'EOF'
            import pstats
            p = pstats.Stats("workspace/tinystories.prof")
            p.sort_stats("tottime").print_stats(30)
            EOF
        \end{lstlisting}
        得到如下结果
        \begin{figure}[H]
            \centering
            \includegraphics[width=\textwidth]{images/QQ20260130-105428.png}
        \end{figure}
        耗时最多的首先是train\_bpe,因为没有添加任何优化和多线程,意料之内,其次是dict类型查找参数耗时比较长
    \end{itemize}

    行内公式示例:$E = mc^2$。

    \section{Latex代码模板}
    在笔记中插入代码是非常常见的需求,特别是 CS 专业的笔记。

    \begin{definition}{梯度下降}
        梯度下降(Gradient Descent)是一种用于优化函数的迭代算法。为了找到函数的局部最小值,我们需要向当前点梯度的\textbf{反方向}迈步。
    \end{definition}

    \begin{theorem}{泰勒公式}
        如果函数 $f(x)$ 在点 $x_0$ 处具有 $n$ 阶导数,则有:
        \begin{equation}
            f(x) = \sum_{n=0}^{\infty} \frac{f^{(n)}(x_0)}{n!} (x-x_0)^n
        \end{equation}
    \end{theorem}

    \begin{lstlisting}[language=Python, caption=Python 示例代码]
    def hello_world():
        print("Hello, LaTeX!")
        return True
    \end{lstlisting}

    \section{重点总结}

    \begin{alert}
        考试重点:请务必记住上述定理的适用条件,不要混淆!
    \end{alert}

\end{document}